%%
%% Optional Insert: HotpotQA Preliminary Results
%% Place this before the "Conclusion" section if results are available.
%%

\subsection{速報:RAGベンチマークへの適用}
geDIGの実用性を検証するため,HotpotQA(distractor setting)を用いた予備実験を行った.
ベースライン(BM25+GPT-4o-mini)と比較して,geDIGエージェントは検索グラフのノード数を約XX\%削減しつつ,同等以上の回答精度(F1: YY\%)を達成した.
これは,geDIGが迷路やTransformerだけでなく,実用的なRAGパイプラインにおいても「必要な情報のみを統合する」フィルタとして機能する可能性を示唆している.
詳細なベンチマーク結果は今後の論文で報告予定である.
