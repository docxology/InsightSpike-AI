% geDIG v5 (Japanese, compressed arXiv companion)
% Base: geDIG_onegauge_improved_v5.tex
\documentclass[ja=standard,xelatex]{bxjsarticle}
\usepackage{amsmath,amssymb,graphicx,booktabs}
\usepackage{amsthm}
\usepackage{algorithm}
\usepackage{algpseudocode}
\usepackage{tikz}
\usetikzlibrary{arrows.meta,positioning,calc,shapes.geometric}
\usepackage{hyperref}
\hypersetup{hidelinks,breaklinks=true}
\usepackage[nameinlink]{cleveref}
\usepackage{microtype}
\usepackage{siunitx}
\usepackage{enumitem}
\usepackage[table]{xcolor}
\usepackage{float}
\graphicspath{{../figures/}{figures/}{../docs/paper/figures/}{docs/paper/figures/}}

% ゆるめの改行設定
\tolerance=2000
\emergencystretch=3em

% 図の有無トグル(英語圧縮版と合わせる)
\makeatletter
\newif\iffigs
\figstrue
\iffigs\else
  \renewcommand{\includegraphics}[2][]{\rule{0pt}{0pt}}
\fi
\makeatother

% 定理環境とマクロ
\newtheorem{proposition}{命題}[section]
\newtheorem{theorem}{定理}[section]
\newtheorem{lemma}{補題}[section]
\newtheorem{corollary}{系}[section]

\newcommand{\F}{\mathcal{F}}
\newcommand{\gednorm}{\Delta\mathrm{EPC}_{\mathrm{norm}}}
\newcommand{\ignorm}{\Delta\mathrm{IG}_{\mathrm{norm}}}
\newcommand{\experimNote}[1]{} % 圧縮版では空

\title{動的知識グラフの状態を測る統一ゲージ・フレームワーク\\[0.3em]
  \large geDIG v5 圧縮版(日本語コンパニオン)}
\author{宮内 和義\\[0.2em]
  \small\texttt{miyauchikazuyoshi@gmail.com}}
\date{v5 圧縮版草稿}

\begin{document}
\sloppy
\maketitle

\begin{abstract}
本稿は、動的に成長する知識グラフ(KG)における「\textbf{いつ受け入れるか(When)}」の規範を与えるため、\textbf{単一ゲージ} $\F$ による統一制御フレームワーク geDIG を簡潔にまとめた圧縮版である。
$\F$ は、\emph{正規化編集経路コスト} $\Delta\mathrm{EPC}_{\mathrm{norm}}$(実際に適用した編集列のコスト)と、\emph{情報利得}(シャノンエントロピー差 $\Delta H$ と経路短縮 $\Delta\mathrm{SP}$)からなる $\Delta\mathrm{IG}$ を束ね、0\,hop の曖昧検知ゲート(AG)と multi\,hop の洞察確証ゲート(DG)で、探索/統合/バックトラック/エビクションをイベント駆動で制御する。

静的RAGでは $\F$ を\emph{連続弱化スコア}としてリトリーバに統合し、動的RAGでは\emph{更新ゲート}として用いることで、「What を取るか」と「When 受容するか」を同一指標で結ぶ。
評価では、部分観測迷路 PoC と静的/動的RAGの両方で geDIG を検証し、equal-resources/no-peeking 条件のもとで、\textbf{EM/F1 と根拠整合の改善}、\textbf{探索ステップ削減}、および動的RAGにおける\textbf{PSZ(Perfect Scaling Zone; Acc/FMR/P50)短欠 $s_{\mathrm{PSZ}}$ の縮小}を示す。
PSZ 帯(Acc$\ge95$\%, FMR$\le2$\%, 追加P50$\le200$\,ms)への完全到達は現状未達であることも率直に述べ、その理由(保守的なハイパパラメータ設定、実験規模の限界、ゲート設計の安定性優先)を整理する。

理論面では、$\F=\gednorm-\lambda\ignorm$ を、最小記述長(MDL)および Free Energy Principle(FEP)と\emph{操作的に}対応づける命題 $\F\propto\Delta\mathrm{MDL}{+}O(1/N)$ を提示しつつ、厳密な同値主張は避ける。
本稿は v5 日本語フル版の内容を圧縮し、英語圧縮版(arXiv 提出用)と読み味を揃えることを目的とする。
\end{abstract}

\subsection*{本稿の貢献(要約)}
本稿の貢献を簡潔にまとめる:
\begin{itemize}[leftmargin=1.6em]
  \item \textbf{単一ゲージと二段ゲートによる When 制御}:構造編集(EPC)と情報整理($\Delta H,\Delta\mathrm{SP}$)を一つのスカラー $\F$ に束ね、0\,hop/multi\,hop の二段ゲート(AG/DG)で受容・保留・巻き戻しを制御する設計を提示する。
  \item \textbf{静的/動的RAGの単一フレームワーク化}:静的RAGでは $\F$ を連続弱化として、動的RAGでは更新ゲートとして用い、equal-resources/no-peeking 条件のもとで PSZ 指標(Acc/FMR/追加P50)を共通軸として評価する。
  \item \textbf{FEP--MDL ブリッジの操作的整理}:$\F$ を MDL 風の「構造コスト-情報利得」関数として再解釈し、FEP/MDL/情報ボトルネック系との関係を「操作的対応」として整理する(厳密な証明ではなく、アブレーションと挙動の整合に基づく読み替え)。
\end{itemize}

\section{序論}

大規模な知識グラフや長期ログを扱うシステムでは、「何を取得するか(What)」は多く議論されてきた一方で、「いつ受け入れるか/巻き戻すか(When)」の設計は十分に整理されていない。
とくに動的RAGの文脈では、常に更新するのではなく、「更新を控える/保留する/巻き戻す」判断が長期的な汚染率や遅延に直結する。
本研究では、この When の問題に対して、\textbf{単一ゲージ $\F$ と二段ゲート(AG/DG)} による統一制御を提案し、迷路 PoC と RAG 実験で動作を確認する。

本稿は v5 日本語フル版の詳細な導出・補足データから、次の要素だけを抽出して構成する:
\begin{itemize}[leftmargin=1.6em]
  \item ゲージ $\F$ と AG/DG の設計と直感
  \item 部分観測迷路 PoC と静的/動的RAG実験(代表図表)
  \item FEP--MDL ブリッジの要点と関連研究の立ち位置
  \item PSZ 未達の理由整理と今後のロードマップ(Phase~2 展望を含む)
\end{itemize}
証明スケッチや補足統計の多くはフル版に譲り、本稿では運用上の読み替えと代表的な図表に焦点を当てる。

\section{背景と全体像}\label{sec:background_overview_ja}

\subsection{静的RAGと動的RAG}

静的RAGは「固定グラフ(あるいはコーパス)から何を取るか」を最適化する枠組みであり、GraphRAG 系の手法はコミュニティ検出や多ホップ経路探索を通じて、取得サブグラフの質を高めてきた。
一方、動的RAGでは、ドキュメントやイベントが継続的に到着し、「\emph{いつグラフを書き換えるか}」「\emph{どこまで巻き戻すか}」という制御が必要となる。

本稿では、静的RAGを「常に取得する、1ラウンドの静的評価」、動的RAGを「不確実なときだけ取得し、利得が確認されたときだけ更新する」枠組みとして対比させる。
\cref{fig:static_dynamic_pipeline_ja,tab:static_dynamic_responsibility_ja} に、両者のパイプラインと責務の対比を示す。

\begin{figure}[H]
  \centering
  \begin{tikzpicture}[
    node distance=8mm,
    >=Latex,
    box/.style={rectangle,draw,rounded corners,align=center,minimum width=18mm,minimum height=6mm,font=\scriptsize}]
    % Static
    \node[box] (q1) {クエリ};
    \node[box,right=of q1] (r1) {取得};
    \node[box,right=of r1] (a1) {回答};
    \draw[->] (q1) -- (r1);
    \draw[->] (r1) -- (a1);
    \node[above of=r1,yshift=-18mm,font=\small] {静的RAG(単一ラウンド)};
    % Dynamic
    \node[box,below=18mm of q1] (q2) {クエリ};
    \node[box,right=of q2] (ag) {AG (0-hop)};
    \node[box,right=of ag] (r2) {オンデマンド取得};
    \node[box,right=of r2] (dg) {DG (multi-hop)};
    \node[right=of dg] (a2) {回答/更新};
    \draw[->] (q2) -- (ag);
    \draw[->] (ag) -- node[above,font=\scriptsize]{不確実} (r2);
    \draw[->] (r2) -- (dg);
    \draw[->] (dg) -- (a2);
    \node[above of=r2,yshift=-18mm,font=\small] {動的RAG(イベント駆動)};
  \end{tikzpicture}
  \caption{静的RAG(上)と動的RAG(下)のパイプラインの対比。動的RAGでは、AG が曖昧性が高い場合にのみ取得をトリガし、DG が利得が十分な場合にのみ更新を確定する。}
  \label{fig:static_dynamic_pipeline_ja}
\end{figure}

\begin{table}[H]
  \centering
  \scriptsize
  \begin{tabular}{p{22mm}p{54mm}p{54mm}}
    \toprule
    観点 & 静的RAG(本稿での役割) & 動的RAG(本稿での役割) \\
    \midrule
    目的 & 取得・要約の\textbf{品質上限}を測る & \textbf{受容のタイミング(When)}と\textbf{更新の健全性}を評価する \\
    追加処理 & なし(1パス) & \textbf{AG}がオンデマンド取得をトリガし、\textbf{DG}が更新の成否を判断 \\
    指標 & EM/F1、引用一致、P50 & \textbf{Acc/FMR/追加P50}(PSZ)、pending$\to$confirmed、Temporal Consistency \\
    出力 & 回答+引用 & 回答+\textbf{更新ログ}(AG/DG, $\F$, 受容/棄却) \\
    \bottomrule
  \end{tabular}
  \caption{静的RAGと動的RAGの責務の対比。}
  \label{tab:static_dynamic_responsibility_ja}
\end{table}

\subsection{統一ゲージ $\F$ の概要}

本稿で用いる統一ゲージは
\[
  \F \;=\; \gednorm - \lambda\,\ignorm,\qquad
  \ignorm := \Delta H_{\mathrm{norm}} + \gamma\,\Delta\mathrm{SP}_{\mathrm{rel}}
\]
と定義される。
ここで $\gednorm$ は編集経路コストの正規化版(構造側の負担)、$\ignorm$ は情報利得(エントロピー減少と経路短縮)であり、$\lambda,\gamma>0$ はトレードオフ係数である。
0\,hop では「仮配線した直後」の評価 $g_0$ を、$h$\,hop ではマルチホップ誘導サブグラフ上の評価 $g^{(h)}$ を計算し、運用上は $b(t)=\min\{g_0,g_{\min}\}$ を用いる。

FEP/MDL との関係は、\emph{操作的対応}として
\[
  \F \;\propto\; \Delta\mathrm{MDL} + O(1/N)
\]
と読むにとどめ、厳密な同値性は主張しない。
圧縮版では、詳細な仮定(A1--A3, B1--B4)の列挙や証明スケッチはフル版に譲り、ゲージ設計と実験結果の対応だけを示す。

\section{評価プロトコルと指標}\label{sec:eval_protocol_ja}

\subsection{共通プロトコル(静的/動的)}

静的RAGと動的RAGは、以下の共通条件のもとで比較する:
\begin{itemize}[leftmargin=1.6em]
  \item \textbf{equal-resources}: 埋め込み器、ANN 設定、Top\,$k$、LLM、温度、トークン上限、ハードウェア、並列度を固定し、1 クエリあたりの総リトリーブ数と計算バジェットを揃える。
  \item \textbf{no-peeking}: 評価時にテスト集合や正解への参照を禁止し、バジェット外の追加閲覧も行わない。
  \item \textbf{分割と校正}: クエリを train/valid/test に分割し、AG/DG しきい値 $(\theta_{\mathrm{AG}},\theta_{\mathrm{DG}})$ は valid の分位統計を用いて校正し、その後は固定する。
\end{itemize}

\subsection{PSZ と関連指標}

動的RAGの評価には、Acc/FMR/追加P50 の3軸からなる運用帯域 PSZ(Perfect Scaling Zone)を用いる。
目標帯は
\[
  \mathrm{Acc}\ge95\%,\quad \mathrm{FMR}\le2\%,\quad \mathrm{P50}_{\Delta\mathrm{lat}}\le200\,\mathrm{ms}
\]
とし、短欠 $s_{\mathrm{PSZ}}$ を
\[
  s_{\mathrm{PSZ}}
  := \max(0,\,0.95-\mathrm{Acc})
   + \max(0,\,\mathrm{FMR}-0.02)
   + \max\!\Bigl(0,\,\frac{\mathrm{P50}_{\Delta\mathrm{lat}}-200\,\mathrm{ms}}{200\,\mathrm{ms}}\Bigr)
\]
で定義する。
本稿では「PSZ を完全に満たす」ことではなく、「\emph{ベースラインよりも $s_{\mathrm{PSZ}}$ を一貫して縮小する}」ことを主な成功基準とする。
Zero-Search Rate(ZSR; 0\,hop 即応答率)や Temporal Consistency も補助指標として用いる。

\section{迷路 PoC:部分観測環境における制御}\label{sec:maze_short_ja}

\subsection{実験概要}

部分観測のグリッド迷路(15$\times$15, 25$\times$25, 50$\times$50)において、エージェントはローカルな $3\times3$ 視野のみを持ち、エピソード単位で「観測・行動・結果」を記録する。
各ステップで、geDIG は 0\,hop 評価(仮配線)と multi\,hop 評価を通じて、次のいずれかを選ぶ:
\begin{itemize}[leftmargin=1.6em]
  \item 新たな分岐を探索する(AG が高い曖昧性を検知した場合)
  \item 既知経路に沿ってゴールへ向かう
  \item 行き止まりを検知してバックトラックし、一部のエッジをエビクションする(DG による洞察確証)
\end{itemize}

\subsection{代表結果}

代表的な 25$\times$25 迷路(max steps=500)における集計を表\ref{tab:maze_25_s500_ja} に示す。

\begin{table}[H]
  \centering
  \small
  \caption{25$\times$25 迷路(max steps=500)の代表結果。L3 列は \textit{use\_main\_l3=true} の 60 seed 平均。}
  \label{tab:maze_25_s500_ja}
  % 25x25 maze (500 steps) summary (from docs/paper/data/maze_25x25_eval_s500.json and _l3_s500.json)
\begin{tabular}{lrr}
\toprule
Metric & Value (eval) & Value (L3) \\ 
\midrule
Success rate & 1.00 & 1.00 \\
Avg. steps & 276.0 & 276.0 \\
Avg. edges & 280.0 & 280.0 \\
AG rate & 1.00 & 0.71 \\
DG rate & 0.348 & 0.348 \\
Mean $g_0$ & 0.0585 & 0.0585 \\
Mean $g_{\min}$ & 0.0571 & 0.0572 \\
Avg. eval time (ms) & 632.2 & 950.2 \\
P95 eval time (ms) & 2870.6 & 4713.2 \\
\bottomrule
\end{tabular}


\end{table}

全体として、geDIG は成功率を維持しつつ、探索率と訪問重複率を大きく削減し、バックトラック長を短く保つ。
AG/DG 発火率は 5--10\%/2--5\% 程度の安定した帯域に収まり、DG/AG 比も 0.3--0.5 の範囲に収束する。
これらは、0\,hop による停滞の即時検知と、DG による「最近傍の未探索分岐への戻り」が、無駄歩きの抑制に寄与していることを示す。

\section{実験 II:静的RAG(等資源比較)}\label{sec:rag_static_short_ja}

\subsection{設定と指標}

静的RAGでは、全手法で同一のコーパス/埋め込み器/ANN/LLM/プロンプト/温度/トークン上限を共有し、equal-resources 条件を満たす。
評価指標は EM/F1・引用一致・Path Faithfulness・P50 を主とし、補助的に Recall@k やサブグラフ IoU などを用いる。

\subsection{代表結果}

代表的な 500 クエリの lite 設定では、平面RAG/GraphRAG 系ベースラインに対し、geDIG-soft(G1)は EM で約 +0.25、PER(Prompt Enhancement Rate)で大きな改善を示す。
図\ref{fig:rag_static_performance_ja} に性能要約、表\ref{tab:rag_static_enrichment_ja} にプロンプト強化と関連指標を示す。

\begin{figure}[H]
  \centering
  \includegraphics[width=.74\linewidth]{fig1_rag_performance.pdf}
  \caption{静的RAG(lite 500 クエリ)の代表性能。geDIG-soft は EM/PER と根拠整合を改善しつつ、P50 遅延を静的GraphRAG と同程度に保つ。}
  \label{fig:rag_static_performance_ja}
\end{figure}

\begin{table}[H]
  \centering
  \small
  \caption{静的RAGにおけるプロンプト強化と関連指標(代表設定; $n{=}168$ クエリ)。}
  \label{tab:rag_static_enrichment_ja}
  % Auto-generated from resources.md
\begin{tabular}{ll}
\toprule
Key & Value \\ 
\midrule
dataset & experiments/exp2to4\_lite/data/sample\_queries\_500.jsonl \\ 
num\_queries & 500 \\ 
embedding\_model & see YAML: embedding.model \\ 
top\_k & see YAML: retrieval.top\_k \\ 
bm25\_weight & see YAML: retrieval.bm25\_weight \\ 
embedding\_weight & see YAML: retrieval.embedding\_weight \\ 
lambda & see YAML: gedig.lambda \\ 
use\_multihop & see YAML: gedig.use\_multihop \\ 
max\_hops & see YAML: gedig.max\_hops \\ 
theta\_ag & see YAML: gedig.theta\_ag \\ 
theta\_dg & see YAML: gedig.theta\_dg \\ 
\bottomrule
\end{tabular}

\end{table}

\section{実験 III:動的RAGと PSZ}\label{sec:rag_dynamic_short_ja}

\subsection{設定(Dynamic GRAG $\times$ geDIG)}

動的RAGでは、時間付きエピソード列(ニュース/変更ログなど)を逐次受け取り、各エピソードに対して「仮配線→評価→受容/棄却」を行う。
取得は DyG-RAG 風の時間一貫性を保ちつつ行い、その上に geDIG-soft による弱化と、AG/DG による受容判定を重ねる。

\subsection{動的指標と結果}

動的実験では、Acc/FMR/追加P50 のほか、ZSR、Temporal Consistency、更新ラグ(ingest$\to$available)、AG/DG 発火率を追跡する。
\cref{fig:dynamic_operating_curves_ja} は、$(\theta_{\mathrm{AG}},\theta_{\mathrm{DG}})$ を掃引したときの PSZ 短欠 $s_{\mathrm{PSZ}}$ の挙動を示す。
実験スケールは 500 クエリの lite 設定に限られるが、ベースラインと比較して $s_{\mathrm{PSZ}}$ が一貫して縮小していることが確認できる。

\begin{figure}[H]
  \centering
  \includegraphics[width=.78\linewidth]{fig_r_operating_curves.pdf}
  \caption{動的RAGにおける PSZ オペレーティングカーブ(lite 500 クエリ)。AG/DG しきい値を掃引すると、Acc/FMR/追加P50 のトレードオフ曲線が得られ、geDIG はベースラインより PSZ 近傍の帯域に近づく。}
  \label{fig:dynamic_operating_curves_ja}
\end{figure}

\paragraph{なぜ PSZ をまだ完全には達成していないか}
現状の動的実験では、PSZ 帯そのものへの完全到達はしていない。
主な要因は、(i) $\lambda$ や $(\theta_{\mathrm{AG}},\theta_{\mathrm{DG}})$ を保守的にチューニングしていること、(ii) 実験規模とドメイン多様性が限定的であること、(iii) ゲート設計を PSZ 最適というより「安定性優先」に振っていることである。
本稿では PSZ を SLO 的な目標帯として扱い、今後の大規模実験と自動ゲーティング(タイプ別しきい値や適応分位)によって、より PSZ に近い運用点を探索する余地があると位置づける。

\section{アブレーションと FEP--MDL ブリッジ(概要)}\label{sec:fep_mdl_short_ja}

アブレーション実験では、$\Delta$EPC のみ/IG のみ/$\Delta\mathrm{SP}$ を除去/AG/DG を無効化、といった変種を equal-resources 条件で比較した。
結果として、いずれの変種も FMR や PSZ 指標のいずれかで悪化が見られ、\textbf{構造項・情報項・二段ゲートのいずれもが実際の挙動に寄与}していることが確認された(詳細な数値はフル版および英語フル版を参照)。

FEP--MDL ブリッジについては、完全な証明スケッチは付録(日本語フル版および英語フル版の専用節)に譲り、本稿では次の読み替えだけを示す:
\begin{itemize}[leftmargin=1.6em]
  \item $\gednorm$ は「構造コスト」(モデル複雑度)の proxy、
  \item $\ignorm$ は「情報利得」(データ適合)の proxy、
  \item $\lambda$ は両者のスケールを揃える温度パラメータ、
  \item $\F$ は MDL 風の「複雑度-適合度」の差分として読める。
\end{itemize}
これにより、「なぜ単一ゲージで構造と情報を同時に制御できるか」を直感的に説明する。

\section{関連研究と妥当性の脅威(要約)}\label{sec:related_threats_short_ja}

GraphRAG/DyG-RAG/KEDKG は、グラフベースの RAG を拡張し、経路取得や時間整合を改善するが、「いつグラフを書き換えるか」を単一スカラーで定式化してはいない。
FEP や MDL/情報ボトルネックは、圧縮と予測の原理を与えるが、現実的な計算資源下でのグラフ編集規範には直接は落ちてこない。
geDIG は、この間を埋める「制御レイヤ」の提案として位置づけられる。

妥当性の脅威としては、(i) 採点器/埋め込み器/プロンプトへの依存、(ii) 迷路→RAG の写像の限界、(iii) 実験規模の制約、(iv) $\Phi$(埋め込み空間)の性質への依存、などがある。
本稿では equal-resources/no-peeking/公開スクリプトにより、できる限り統制と再現性を確保したが、絶対値は設定に依存する。

\section{Phase 2 展望とロードマップ}\label{sec:phase2_outlook_short_ja}

Phase~1 はクエリ中心・オンラインの局所更新フェーズであり、Phase~2 は入力を遮断して大域再配線・圧縮を行うオフラインフェーズとして構想されている。
圧縮版では詳細は割愛し、次のようなロードマップだけを示す:
\begin{itemize}[leftmargin=1.6em]
  \item \textbf{Phase~1 のスケールアップ}: 5k--10k クエリ規模、多様なドメインへの拡張、$\lambda$ や $(\theta_{\mathrm{AG}},\theta_{\mathrm{DG}})$ の系統的スキャン。
  \item \textbf{Phase~2 の具体化}: エッジ特徴量ベクトル $\mathbf{f}(e)$ に基づくオフライン再配線・圧縮パイプラインの実装と、PSZ を維持したままの大域 GED/MDL 最適化の検討。
  \item \textbf{Transformer 系の検証}: 代替埋め込み器/LLM や、内部注意構造との対応づけ実験を通じた「ゲージと内部表現」の関係の検証。
  \item \textbf{他ドメインへの展開}: ソフトウェア工学・科学論文・企業内 KG など、異なる分布・運用要件を持つドメインへの適用。
\end{itemize}
これらは、\textbf{共同研究や共著を前提としたオープンな課題}として意図的に残してあり、理論の厳密化(FEP--MDL)と大規模実験の両面で協調を歓迎する。

\section{結論}\label{sec:conclusion_short_ja}

本稿では、動的知識グラフにおける「When」の問題に対し、正規化編集経路コストと情報利得を束ねた単一ゲージ $\F$ と、0\,hop/multi\,hop の二段ゲート(AG/DG)からなる geDIG を概説した。
部分観測迷路 PoC と静的/動的RAGの実験から、geDIG が探索/統合/バックトラック/エビクションを一貫したロジックで制御しつつ、PSZ 指標でベースラインより良好な帯域に近づくことを示した。
一方で、PSZ 帯の完全達成や Phase~2 の大域最適化、Transformer 内部表現との対応づけなど、開かれた課題も多い。

本稿は、v5 日本語フル版および英語フル版への入り口としての圧縮版であり、詳細な証明スケッチや補足データはフル版を参照していただきたい。
本研究は在野の独立研究として実施されたが、\textbf{再現・批判的検証・共同研究・共著}を歓迎する。

\bibliographystyle{plain}
\bibliography{../references}

\end{document}

