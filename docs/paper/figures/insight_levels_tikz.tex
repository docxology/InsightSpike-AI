% 図1: 閃きの階層レベル
% Level 3 → Level 2 → Level 1 の階層構造

\begin{figure}[htbp]
\centering
\begin{tikzpicture}[
    level/.style={rectangle, draw, rounded corners, align=center, minimum width=50mm, minimum height=12mm, font=\small},
    arrow/.style={->, >=Latex, thick},
    note/.style={font=\scriptsize, text=gray}
]

% Level 3: Isomorphism Discovery
\node[level, fill=blue!20] (L3) at (0, 4) {
    \textbf{Level 3: 同型発見}\\
    $T^* = \arg\min_T \mathrm{GED}(T(G_1), G_2)$
};
\node[note, right=5mm of L3] {Einstein: ローレンツ変換};

% Level 2: Analogy Detection
\node[level, fill=green!20] (L2) at (0, 2) {
    \textbf{Level 2: アナロジー検出}\\
    $\mathrm{SS}(G_1, G_2) > \theta$
};
\node[note, right=5mm of L2] {Bohr: 太陽系 $\approx$ 原子};

% Level 1: Pattern Matching
\node[level, fill=yellow!20] (L1) at (0, 0) {
    \textbf{Level 1: パターンマッチ}\\
    $\mathrm{sim}(a, b) = \cos(\phi(a), \phi(b))$
};
\node[note, right=5mm of L1] {要素の類似性検出};

% Arrows
\draw[arrow] (L1) -- (L2);
\draw[arrow] (L2) -- (L3);

% Labels on arrows
\node[note] at (-1.5, 1) {構造化};
\node[note] at (-1.5, 3) {変換発見};

% geDIG coverage
\draw[dashed, thick, blue!60] (-3.2, -0.8) rectangle (3.2, 2.8);
\node[font=\footnotesize, text=blue!60] at (2.5, -0.5) {geDIG Phase 1};

\end{tikzpicture}
\caption{閃きの三階層モデル。Level 1は要素間の類似性、Level 2は構造間のアナロジー、Level 3は構造を同型にする変換の発見を表す。現在のgeDIGはLevel 1-2をカバーし、Level 3は将来課題である。}
\label{fig:insight_levels}
\end{figure}


% 図2: 分子設計AIとの対応
% Scaffold Hopping と Theory Unification の並列比較

\begin{figure}[htbp]
\centering
\begin{tikzpicture}[
    box/.style={rectangle, draw, rounded corners, align=center, minimum width=28mm, minimum height=15mm, font=\small},
    arrow/.style={->, >=Latex, thick},
    label/.style={font=\footnotesize, text=gray}
]

% Left side: Molecular Design
\node[font=\bfseries] at (-3, 4.5) {分子設計AI};

\node[box, fill=red!10] (mol1) at (-3, 3) {分子A\\(元の骨格)};
\node[box, fill=red!20] (mol2) at (-3, 0.5) {分子B\\(新しい骨格)};
\draw[arrow, red!70] (mol1) -- node[left, label] {GED\\(原子追加/削除)} (mol2);

\node[label] at (-3, -1) {\textbf{同じ薬効}};
\node[font=\scriptsize] at (-3, -1.5) {Scaffold Hopping};

% Right side: Knowledge Discovery
\node[font=\bfseries] at (3, 4.5) {geDIG (GED-Insight)};

\node[box, fill=blue!10] (th1) at (3, 3) {理論A\\(既存の構造)};
\node[box, fill=blue!20] (th2) at (3, 0.5) {理論B\\(新しい構造)};
\draw[arrow, blue!70] (th1) -- node[right, label] {GED\\(概念追加/削除)} (th2);

\node[label] at (3, -1) {\textbf{同じ説明力}};
\node[font=\scriptsize] at (3, -1.5) {Theory Unification};

% Central correspondence
\draw[<->, dashed, thick, gray] (-0.5, 3) -- node[above, font=\footnotesize] {数学的同型} (0.5, 3);
\draw[<->, dashed, thick, gray] (-0.5, 0.5) -- node[above, font=\footnotesize] {} (0.5, 0.5);

% Bottom summary
\node[draw, rounded corners, fill=gray!10, minimum width=80mm, minimum height=10mm, font=\small] at (0, -2.5) {
    分子グラフのトポロジカル探索 $\Leftrightarrow$ 知識グラフのトポロジカル探索
};

\end{tikzpicture}
\caption{分子設計AIとgeDIGの数学的対応。創薬における Scaffold Hopping(同じ薬効を持つ異構造分子の発見)と、知識における Theory Unification(同じ説明力を持つ異構造理論の発見)は、GED最小化という同一の数学的枠組みで記述できる。}
\label{fig:molecular_correspondence}
\end{figure}


% 図3: アインシュタインの相対論の構造的理解
% 電磁気学と古典力学の統合

\begin{figure}[htbp]
\centering
\begin{tikzpicture}[
    theory/.style={ellipse, draw, align=center, minimum width=35mm, minimum height=20mm, font=\small},
    transform/.style={rectangle, draw, rounded corners, fill=yellow!30, align=center, minimum width=40mm, font=\small},
    arrow/.style={->, >=Latex, thick}
]

% Two conflicting theories
\node[theory, fill=red!15] (em) at (-3, 2) {電磁気学\\(Maxwell方程式)};
\node[theory, fill=blue!15] (cm) at (3, 2) {古典力学\\(Newton方程式)};

% Conflict
\draw[<->, thick, red, dashed] (em) -- node[above, font=\footnotesize, text=red] {矛盾} (cm);

% Transformation (Insight)
\node[transform] (lorentz) at (0, -0.5) {\textbf{ローレンツ変換}\\$T^* = \arg\min_T \mathrm{GED}(T(G_1), G_2)$};

% Arrows to transformation
\draw[arrow] (em) -- (lorentz);
\draw[arrow] (cm) -- (lorentz);

% Unified theory
\node[theory, fill=green!20] (sr) at (0, -3) {特殊相対論\\(統一された時空構造)};

% Arrow from transformation to unified
\draw[arrow, thick] (lorentz) -- node[right, font=\footnotesize] {同型化} (sr);

% Labels
\node[font=\scriptsize, text=gray] at (-4.5, 0) {入力: 2つの構造};
\node[font=\scriptsize, text=gray] at (4.5, -0.5) {発見: 変換$T^*$};
\node[font=\scriptsize, text=gray] at (0, -4.2) {出力: 統一理論};

% Box around the insight
\draw[dashed, thick, orange] (-2.5, -1.2) rectangle (2.5, 0.2);
\node[font=\footnotesize, text=orange] at (2, 0.5) {\textbf{= 閃きの実体}};

\end{tikzpicture}
\caption{アインシュタインの相対論の構造的理解。電磁気学と古典力学という矛盾する2つの理論構造を、ローレンツ変換という最小編集操作で同型にした。この変換の発見こそが「閃き」の計算的実体である。}
\label{fig:einstein_insight}
\end{figure}


% 図4: geDIGのロードマップ(拡張版)
% Phase 1 → ... → Level 3 Insight Discovery

\begin{figure}[htbp]
\centering
\begin{tikzpicture}[
    node distance=10mm,
    phase/.style={rectangle, draw, rounded corners, align=center, minimum width=22mm, minimum height=12mm, font=\scriptsize},
    done/.style={phase, fill=green!20},
    current/.style={phase, fill=yellow!20},
    future/.style={phase, fill=gray!10},
    arrow/.style={->, >=Latex}
]

% Row 1: Current achievements
\node[done] (p1) {Phase 1\\When制御\\(Maze/RAG)};
\node[done, right=of p1] (ss) {構造類似度\\Level 2\\(F1 +60\%)};
\node[current, right=of ss] (embed) {埋め込み統一\\Phase 6};
\node[future, right=of embed] (offline) {オフライン再配線\\Phase 2};

% Row 2: Future
\node[future, below=15mm of p1] (attn) {注意グラフ\\解析};
\node[future, below=15mm of ss] (layer) {層ゲート\\制御};
\node[future, below=15mm of embed] (iso) {\textbf{Level 3}\\同型発見};
\node[future, below=15mm of offline] (world) {世界モデル\\統合};

% Arrows
\draw[arrow] (p1) -- (ss);
\draw[arrow] (ss) -- (embed);
\draw[arrow] (embed) -- (offline);

\draw[arrow] (p1) -- (attn);
\draw[arrow] (ss) -- (layer);
\draw[arrow] (embed) -- (iso);
\draw[arrow] (offline) -- (world);

\draw[arrow] (attn) -- (layer);
\draw[arrow] (layer) -- (iso);
\draw[arrow] (iso) -- (world);

% Labels
\node[font=\footnotesize, text=gray] at (-2, 0.8) {達成済み};
\node[font=\footnotesize, text=gray] at (-2, -2) {将来課題};

% Highlight Level 3
\draw[thick, orange, dashed] ([xshift=-3mm, yshift=3mm]iso.north west) rectangle ([xshift=3mm, yshift=-3mm]iso.south east);
\node[font=\scriptsize, text=orange] at ([yshift=-8mm]iso.south) {本論文の最終目標};

\end{tikzpicture}
\caption{geDIGの発展ロードマップ。Phase 1(When制御)と構造類似度(Level 2アナロジー検出)は達成済み。Level 3(同型発見)の実現が本論文の最終目標であり、「理論を発見するAI」への道を開く。}
\label{fig:gedig_roadmap_extended}
\end{figure}
