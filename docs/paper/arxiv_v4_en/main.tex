% geDIG v4 (English Draft)
% arXiv-friendly preamble
\pdfoutput=1
\documentclass[12pt]{article}
\usepackage[utf8]{inputenc}
\usepackage[T1]{fontenc}
\usepackage{lmodern}
\usepackage{amsmath,amssymb,graphicx,booktabs}
\usepackage{amsthm}
\usepackage{algorithm}
\usepackage{algpseudocode}
\usepackage{tikz}
\usetikzlibrary{arrows.meta,positioning,calc,shapes.geometric}
\usepackage[hidelinks]{hyperref}
\usepackage[nameinlink]{cleveref}
\usepackage{microtype}
\usepackage{siunitx}
\usepackage{enumitem}
\usepackage[table]{xcolor}
\usepackage{float}
\usepackage[a4paper,margin=1in]{geometry}
\graphicspath{{figures/}{docs/paper/figures/}}

% Build switch: ignore missing figures when compiling
\makeatletter
\newif\iffigs
\figsfalse
\iffigs\else
  \renewcommand{\includegraphics}[2][]{\rule{0pt}{0pt}}
\fi
\makeatother

% Theorems
\newtheorem{proposition}{Proposition}[section]
\newtheorem{theorem}{Theorem}[section]
\newtheorem{lemma}{Lemma}[section]
\newtheorem{corollary}{Corollary}[section]

% Macros
\newcommand{\F}{\mathcal{F}}
\newcommand{\gednorm}{\Delta\mathrm{EPC}_{\mathrm{norm}}}
\newcommand{\ignorm}{\Delta\mathrm{IG}_{\mathrm{norm}}}
\newcommand{\experimNote}[1]{\textcolor{red}{\textbf{[TBD: #1]}}}

\title{geDIG: A One-Gauge Framework for Controlling Dynamic Knowledge Graphs}
\author{Kazuyoshi Miyauchi\\ \small\texttt{miyauchikazuyoshi@gmail.com}}
\date{Draft (v4, English)}

\begin{document}
\maketitle

\begin{abstract}
We propose geDIG, a single-gauge ($\F$) control framework for dynamic knowledge graphs that unifies structure cost (normalized edit path cost, $\Delta$EPC) and information gain (entropy decrease and path shortening, $\Delta$IG). The two-stage gating (AG: attention, DG: decision) drives acceptance/rejection/exploration/backtrack/eviction in an event-driven manner. The design cleanly separates 0-hop novelty/error detection (FEP-side) from multi-hop compression/shortcuts (MDL-side), achieving practical anytime operation. We present results-first summaries for four experiments: (I) PoC on partial-observation mazes, (II) static RAG baselines (flat vs GraphRAG/GNN vs Graph Transformer vs geDIG-soft), (III) dynamic GRAG with PSZ (Perfect Scaling Zone), and (IV) insight-vector alignment. Finally, we formulate an operational FEP–MDL bridge and provide a thermodynamic reading (free energy) of $\F$.
\end{abstract}

\section{Introduction}
We study when and how to accept, connect, and reuse new knowledge episodes in a dynamic knowledge graph (KG). Our core hypothesis: a single numerical gauge $\F$ can reliably drive both learning (curation) and inference (retrieval/use).

\paragraph{One-Gauge and Two-Stage Gates}
We define
\begin{equation}
  \F = \gednorm - \lambda\,\ignorm,\quad \ignorm = \Delta H_{\mathrm{norm}} + \gamma\,\Delta\mathrm{SP}_{\mathrm{rel}},
\end{equation}
where $\lambda$ sets the information temperature and $\gamma$ balances entropy vs path-efficiency. AG (attention) triggers on high 0-hop novelty/error; DG (decision) commits only when multi-hop gain is confirmed (shortcuts/compression).

\paragraph{Contributions (short)}
\begin{itemize}[leftmargin=1.2em]
  \item Results-first, unified control: $\F$ and two-stage gates for online acceptance/search/eviction.
  \item Static and dynamic RAG: clean split; dynamic metrics (PSZ, FMR) isolated in the Dynamic chapter.
  \item Operational FEP–MDL bridge and a free-energy reading of $\F$ (engineering, not identity).
\end{itemize}

\section{Design: One Gauge and Two-Stage Gating}
\subsection{0-hop vs Multi-hop: FEP and MDL}
0-hop evaluates draft wiring at the query hub (novelty/error; FEP-side), while multi-hop evaluates shortcuts/compression on induced subgraphs (MDL-side). Let $g_0 = \Delta\mathrm{EPC}_{\mathrm{norm}} - \lambda\,\Delta H_{\mathrm{norm}}$ and $g_{\min}=\min_h\{\Delta\mathrm{EPC}_{\mathrm{norm}}-\lambda(\Delta H_{\mathrm{norm}}+\gamma\,\Delta\mathrm{SP}_{\mathrm{rel}}^{(h)})\}$. AG fires if $g_0>\theta_{\mathrm{AG}}$; DG fires if $\min\{g_0,g_{\min}\}\le\theta_{\mathrm{DG}}$.

\subsection{Thermodynamic Reading (Metaphor)}
We can read $\F$ as an operational free energy:
\begin{equation}
  U := \Delta\mathrm{EPC}_{\mathrm{norm}} - \lambda\,\gamma\,\Delta\mathrm{SP}_{\mathrm{rel}},\quad S := \Delta H_{\mathrm{norm}},\quad F := U - \lambda S,\label{eq:free_energy_en}
\end{equation}
so $\F$ is isomorphic to $F$ by term rearrangement. The coefficient $\lambda$ plays the role of information temperature.

\section{Experiment I: Maze PoC (results-first)}
\paragraph{Summary} geDIG achieves large reductions in exploration ratio and revisit rate, with short backtracks and near-immediate dead-end detection. Example (25×25): \experimNote{exploration 0.38, revisit 1.28, backtrack 4.3, detection 0.8, success 100\%}.

\paragraph{Metrics} Primary: exploration ratio (unique/total), revisit (steps/unique), avg backtrack (AG→DG), dead-end detection delay, success rate. Secondary: Regret, SPL.

\paragraph{Success Criteria} Necessary: success\,$\ge$\,95\%, AG 5–10\%, DG 2–5\%, DG/AG 30–50\%, threshold stability (train/val within 2\%). Sufficient: exploration\,$\le$\,0.40, revisit\,$\le$\,1.5, backtrack\,$\le$\,5, detection\,$\le$\,1 with significance vs Greedy Novelty (Welch+Bonferroni, $p{<}0.01$, $d{>}0.5$). Diagnostic: Regret median\,$\le$\,+5, SPL mean\,$\ge$\,0.85.

\paragraph{Baselines (same conditions)} Greedy Novelty, $\varepsilon$-greedy, UCB1-like, Partially-Observed A*, and ablations (EPC-only / IG-only / no AG/DG / 0-hop only). Dijkstra/A* used as upper-bound diagnostics.

\section{Experiment II: RAG Baselines (static only)}
\paragraph{Summary} Under equal-resources, geDIG-soft (G1) improves EM/F1 by \experimNote{+2–5pt} over GT-RAG (B2), increases citation/path faithfulness by \experimNote{+5–10pt}, with comparable P50/P95 latency.

\paragraph{Baselines} B0: Flat RAG (SBERT, HNSW, Top-k), B1: GraphRAG (GNN), B2: Graph Transformer, G1: B2 + geDIG-soft (sigmoid$\,(\tau\F)$ for weighting/pruning/ordering). Static-only here; dynamic is in Experiment III.

\paragraph{Dataset and Protocol} 50 domains (mix of single-domain, cross-domain 2/3-hop, analogical). Sources: HotpotQA/2Wiki + curated. Equal-resources table (embedder/ANN/Top-k/LLM/temp/tokens/HW/parallelism/measurement) fixed across methods. No-peeking: train (burn-in for thresholds) / val / test split; thresholds fixed on val.

\section{Experiment III: Dynamic GRAG × geDIG}
\paragraph{Summary} With geDIG-soft applied consistently to retrieval/integration/summarization (G2), Temporal Consistency improves by \experimNote{+5–10pt}, update lag remains comparable or lower, KG contamination (FMR) decreases, and PSZ points emerge.

\paragraph{Dynamic Metrics} Temporal Consistency, update lag (ingest→available), KG contamination rate (FMR, rolling), 0-hop rejection, AG/DG rates. PSZ: Acc\,$\ge$\,95\%, FMR\,$\le$\,2\%, extra P50\,$\le$\,200ms.

\paragraph{Time-Series and Operating Curves} Plot $\Delta$EPC/$\Delta H$/\,$\Delta$SP/$\F$ with acceptance time-series (pending→confirmed, C-value), and Operating Curves (Acc–FMR–Latency) with PSZ band.

\section{Experiment IV: Insight-Vector Alignment}
\paragraph{Summary} Readout vectors from DG-confirmed subgraphs align with LLM answer embeddings: \experimNote{$\Delta s{=}{+}0.2x$, $p{<}0.0x$, $N{=}200$}. Baselines: random, Top-k, threshold, AG-selected.

\section{FEP–MDL Bridge (operational proposition)}
\paragraph{Definition} We call an operational correspondence a relation that (i) is proportional (not identical), (ii) has a bounded residual $O(1/N)$ under assumptions, and (iii) yields testable predictions. Under mild assumptions (normalization, bounded horizon, decomposable edits, stable entropy estimation), $\F\propto\Delta\mathrm{MDL}+O(1/N)$. The coefficient $\lambda\approx c_D/c_M$ anchors scales.

\paragraph{Implications} A single control signal justifies simultaneous control of structure edits and inference. EPC on the structure side and IG on the information side avoid double counting. Ablations corroborate the roles of $\Delta H$ and $\Delta$SP.

\section{Conclusion}
We presented geDIG, a one-gauge control framework with two-stage gates, covering PoC (maze), static RAG baselines, dynamic GRAG (PSZ), and insight alignment, and provided an operational FEP–MDL bridge (free-energy reading). Future work includes Phase~2 (offline rewiring) and large-scale evaluations.

\vspace{0.5em}
\bibliographystyle{plain}
\bibliography{references}

\end{document}
